\documentclass[UTF8]{ctexart}

\usepackage{array,tabularx}
\usepackage{booktabs}
\usepackage{multirow}
\usepackage{amsmath}
\usepackage{amssymb}
\usepackage{makecell}
\usepackage{graphicx}
\usepackage{subcaption}

%opening
\title{}
\author{}

\begin{document}

	\maketitle

	\begin{abstract}

	\end{abstract}


	\section{}


	\fbox{
		\begin{minipage}{15em}%
			这是一个垂直盒子的测试。
			\footnote{脚注来自 minipage。}
		\end{minipage}
	}

	The Pythagorean theorem is:
	\begin{equation}
		a^2 + b^2 = c^2 \label{pythagorean}
	\end{equation}
	Equation \eqref{pythagorean} is
	called `Gougu theorem' in Chinese.

	\begin{equation*}
		a^2 + b^2 = c^2
	\end{equation*}
	For short:
	\[a^2 + b^2 = c^2 \]
	Or if you like the long one:
	\begin{displaymath}
		a^2 + b^2 = c^2
	\end{displaymath}

	$x^{2} \geq 0 \qquad
	\text{for \textbf{all} }
	x\in\mathbb{R}$

	${a}^{2}+{b}^{2}={c}^{2}$

	\begin{enumerate}
		\item some thing.
		\begin{enumerate}
			\item A nested item.\label{itref}
			\item[*] A starred item.
		\end{enumerate}
		\item Reference(\ref{itref}).
	\end{enumerate}

	\renewcommand{\labelitemi}{\ddag}
	\renewcommand{\labelitemii}{\dag}
	\begin{itemize}
		\item First item
		\begin{itemize}
			\item Subitem
			\item Subitem
		\end{itemize}
		\item Second item
	\end{itemize}

	\renewcommand{\labelenumi}%
	{\Alph{enumi}>}
	\begin{enumerate}
		\item First item
		\item Second item
	\end{enumerate}

	\begin{center}
		Centered text using a
		\verb|center| environment.
	\end{center}

% 会引入额外的行间距

	\begin{flushleft}
		Left-aligned text using a
		\verb|flushleft| environment.
	\end{flushleft}

	\begin{flushright}
		Right-aligned text using a
		\verb|flushright| environment.
	\end{flushright}

% 不会引入额外的行间距

	\centering
	Centered text paragraph.

	\raggedright
	Left-aligned text paragraph.

	\raggedleft
	Right-aligned text paragraph.

% 单行引用

	Francis Bacon says:
	\begin{quote}
		Knowledge is power.
	\end{quote}

% 《多行引用
	《 木兰诗》:

	\begin{quotation}
		万里赴戎机,关山度若飞。

		朔气传金柝,寒光照铁衣。

		将军百战死,壮士十年归。

		归来见天子,天子坐明堂。

		策勋十二转,赏赐百千强。
	\end{quotation}

% 代码块

	\begin{verbatim}
	#include <iostream>
	int main()
	{
		std::cout << "Hello, world!"
		<< std::endl;
		return 0;
	}
	\end{verbatim}

% 代码块显示空格,应该用于调试

	\begin{verbatim*}
		for (int i=0; i<4; ++i)
		printf("Number %d\n",i);
	\end{verbatim*}

% \verb⟨delim⟩⟨code⟩⟨delim⟩

	\verb|\LaTeX| \\
	\verb+(a || b)+ \verb*+(a || b)+

% 为何 t 和 b 的控制似乎相反了???

	\begin{tabular}{|c|}
		center-\\ aligned \\
	\end{tabular},

	\begin{tabular}[t]{|c|}
		top-\\ aligned \\
	\end{tabular},

	\begin{tabular}[b]{|c|}
		bottom-\\ aligned\\
	\end{tabular} tabulars.

% table control

	\begin{tabular}{lcr|p{6em}}
		\hline
		left & center & right
		& par box with fixed width \\
		L
		& C
		& R
		& P \\
		\hline
	\end{tabular}

	\begin{tabular}{@{} r@{:}lr @{}}
		\hline
		1 & 1 & one    \\
		11 & 3 & eleven \\
		\hline
	\end{tabular}


% \usepackage{array}
	\begin{tabular}%
	{>{\centering\arraybackslash}p{9em}}
		\hline
		Some center-aligned long text. \\
		\hline
	\end{tabular}

% \usepackage{array}
	\newcommand\txt{a b c d e f g h i}
	\begin{tabular}{cp{2em}m{2em}b{2em}}
		\hline
		pos & \txt & \txt & \txt \\
		\hline
	\end{tabular}

% control columne width: with bug of central 2 columns alignments
	\begin{tabular*}{14em}%
	{@{\extracolsep{\fill}}|c|c|c|c|}
		\hline
		A & B & C & D \\ \hline
		a  & b & c & d \\ \hline
	\end{tabular*}


% 多个 X 列格式平均分配列宽
% \usepackage{array,tabularx} 
	\begin{tabularx}{14em}%
	{|*{4}{>{\centering\arraybackslash}X|}}
		\hline
		A & B & C & D \\ \hline
		a  & b & c & d \\ \hline
	\end{tabularx}

% 跨列横线

	\begin{tabular}{|c|c|c|}
		\hline
		4 & 9 & 2 \\ \cline{2-3}
		3  & 5 & 7 \\ \cline{1-1}
		8  & 1 & 6 \\ \hline
	\end{tabular}

% 三线表
% \usepackage{booktabs}
	\begin{tabular}{cccc}
		\toprule
		& \multicolumn{3}{c}{Numbers} \\
		\cmidrule{2-4}
		& 1 & 2  & 3   \\
		\midrule
		Alphabet & A & B  & C   \\
		Roman
		& I & II & III \\
		\bottomrule
	\end{tabular}

% 跨越多行

% \usepackage{multirow}
	\begin{tabular}{ccc}
		\hline
		\multirow{2}{*}{Item} &
		\multicolumn{2}{c}{Value} \\
		\cline{2-3}
		& First & Second \\ \hline
		A & 1
		& 2 \\ \hline
	\end{tabular}

% 用 \multicolumn 命令配合 @{} 格式把单元格的额外边距去掉

	\begin{tabular}{|c|c|c|}
		\hline
		a & b & c \\ \hline
		a & \multicolumn{1}{@{}c@{}|}
		{\begin{tabular}{c|c}
			 e & f \\ \hline
			 e & f \\
		\end{tabular}}
		& c \\ \hline
		a  & b & c \\ \hline
	\end{tabular}

% 需要安装这个包
	\begin{tabular}{|c|c|}
		\hline
		a & \makecell{d1 \\ d2}  \\
		\hline
		b  & c            \\
		\hline
	\end{tabular}

% 行距控制
	\renewcommand\arraystretch{1.8}
	\begin{tabular}{|c|}
		\hline
		Really loose \\ \hline
		tabular rows. \\ \hline
	\end{tabular}

% 行距控制方式 2

	\begin{tabular}{c}
		\hline
		Head lines \\[6pt]
		tabular lines \\
		tabular lines \\ \hline
	\end{tabular}

	1

	2

	3

	4

	5

	6

	7

	8

	9

% 插入图片
	\begin{figure}[htb]
		\centering{\includegraphics[scale=.7]{figures/miku-01.png}}
		\caption{This is a cute miku.}
		\label{fig}
	\end{figure}

% === mbox 只有一行的盒子, makebox 控制粒度更细

	\mbox{Test some words.}\\
	|\makebox[10em]{Test some words.}|\\
	|\makebox[10em][l]{Test some words.}|\\
	|\makebox[10em][r]{Test some words.}|\\
	|\makebox[10em][s]{Test some words.}|

% fbox 边框盒子

	\fbox{Test some words.}\\
	\framebox[10em][r]{Test some words.}

% 通过 \setlength 命令调节边框的宽度 \fboxrule 和内边距 \fboxsep
	\framebox[10em][r]{Test box}\\[1ex]
	\setlength{\fboxrule}{1.6pt}
	\setlength{\fboxsep}{1em}
	\framebox[10em][r]{Test box}

% 垂直盒子

	三字经:\parbox[t]{3em}%
	{人之初 性本善 性相近 习相远}
	\quad
	千字文:
	\begin{minipage}[b][8ex][t]{4em}
		天地玄黄 宇宙洪荒
	\end{minipage}

	以 markdown 写内容,以 latex 或者 web 界面框架来排版

% 垂直盒子与迷你页

	\fbox{\begin{minipage}{15em}%
			  这是一个垂直盒子的测试。
			  \footnote{脚注来自 minipage。}
	\end{minipage}}

% LATEX 预定义了两类浮动体环境 figure 和 table
% 图表等浮动体提供了 \caption 命令加标题,并且自动给浮动体编号

% 并排和子图表 (缺乏标题)

	\begin{figure}[htbp]
		\centering
		{\includegraphics[scale=.3]{figures/miku-01.png}}
		\qquad
		{\includegraphics[scale=.4]{figures/miku-01.png}} \\[2pt]
		{\includegraphics[scale=.7]{figures/miku-01.png}}
		\caption{...}
	\end{figure}

% with caption

	\begin{figure}[htbp]
		\begin{minipage}[b]    {15em}
			\centering
			{\includegraphics[scale=0.4]{figures/miku-01.png}}
			\caption{miku-1}
		\end{minipage}
		\qquad
		\begin{minipage}[b]{15em}
			\centering
			{\includegraphics[scale=0.4]{figures/miku-01.png}}
			\caption{miku-2}
		\end{minipage}
	\end{figure}

% with subcaption

	\begin{figure}[htbp]
		\centering
		\begin{subfigure}{0.45\textwidth}
			\centering
			\includegraphics[scale=0.4]{figures/miku-01.png}
			\subcaption{f1}
		\end{subfigure}
		\qquad
		\begin{subfigure}{0.45\textwidth}
			\centering
			\includegraphics[scale=0.4]{figures/miku-01.png}
			\subcaption{f2}
		\end{subfigure}
	\end{figure}

\end{document}
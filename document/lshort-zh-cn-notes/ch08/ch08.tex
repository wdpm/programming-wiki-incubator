\documentclass[UTF8]{ctexart}

\usepackage{listings}

%opening
\title{}
\author{}

\begin{document}

% 【自定义命令】

	\newcommand{\tnss}{The not so Short Introduction to \LaTeXe}

	This is ``\tnss'' \ldots{} ``\tnss''

	\newcommand{\txsit}[1]{This is the
	\emph{#1} Short Introduction
	to \LaTeXe}
% in the document body:
	\begin{itemize}
		\item \txsit{not so}
		\item \txsit{very}
	\end{itemize}

% - 使用 \renewcommand 命令 覆盖式声明
% - 使用 \providecommand 命令是一种比较理想的方案:在命令未定义时,它
% 相当于 \newcommand;在命令已定义时,沿用已有的定义

% 自定义环境

	\newenvironment{king}
	{\rule{1ex}{1ex}%
	\hspace{\stretch{1}}}
	{\hspace{\stretch{1}}%
	\rule{1ex}{1ex}}
	\begin{king}
		My humble subjects \ldots
	\end{king}

% 同理:\renewenvironment

%=============

% xparse 宏包(LATEX 2020-10-01 版本之后,xparse 宏包已集成在了格式之中,不需要显式调用)
% \NewDocumentCommand\⟨name⟩{⟨arg spec⟩}{⟨definition⟩}
% \NewDocumentEnvironment{⟨name⟩}{⟨arg spec⟩}{⟨before⟩}{⟨after⟩}

% 百分号用于注释掉不必要的空格和换行符
	\NewDocumentCommand\hello{om}
	{%
		\IfNoValueTF{#1}%
		{Hello, #2!}%
		{Hello, #1 and #2!}%
	}
	\hello{Alice}
	\hello[Bob]{Alice}

	\NewDocumentCommand\hereis{sm}
	{Here is \IfBooleanTF{#1}{an}{a} #2.}
	\hereis{banana}
	\hereis*{apple}

	\NewDocumentEnvironment {envstar} {s}
	{\IfBooleanTF {#1} {star} {no star}} {}
	\begin{envstar}
		*
	\end{envstar}

% 定义命令

	\begin{lstlisting}
\NewDocumentCommand
\NewDocumentEnvironment

\RenewDocumentCommand
\RenewDocumentEnvironment

\ProvideDocumentCommand
\ProvideDocumentEnvironment

这个有点差别,已定义 => 覆盖, 未定义 => 定义
\DeclareDocumentCommand
\DeclareDocumentEnvironment
	\end{lstlisting}

% 编写自己的宏包和文档类

	counter

	在 article 文档类里 part 为 0,section 为 1,依此类推;

	在 report / book 文档类里 part 为 -1,chapter 为 0,section 为 1,等等

	secnumdepth 计数器在 article 文档类里默认为 3(subsubsection 一级);在 report 和 book
	文档类里默认为 2(subsection 一级)。

	tocdepth 计数器控制目录的深度,如果章节的层级大于 tocdepth,那么章节将不会自动写
	入目录项。默认值同 secnumdepth。

% LATEX 可定制的一些命令和参数

	- 标题名称 / 前后缀等。

	- 长度。

\end{document}
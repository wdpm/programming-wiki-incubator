\documentclass[UTF8]{ctexart}

\usepackage{fontspec}
\usepackage{ulem}

\setCJKmainfont{SimSun}[BoldFont=SimHei, ItalicFont=KaiTi]

% 在导言区使用此代码
\usepackage{fancyhdr}
\pagestyle{fancy}
% LATEX 默认将页眉的内容都转为大写字母。如果需要保持字母的大小写,执行下面两行
%\renewcommand{\chaptermark}[1]{\markboth{#1}{}}
%\renewcommand{\sectionmark}[1]{\markright{\thesection\ #1}}
% 使用 \fancyhf{} 来清空页眉页脚的设置
\fancyhf{}
\fancyfoot[C]{\bfseries\thepage}
% 奇数页,页眉内容左对齐
\fancyhead[LO]{\bfseries\rightmark}
% 偶数页,页眉内容靠右对齐
\fancyhead[RE]{\bfseries\leftmark}
\renewcommand{\headrulewidth}{0.4pt} % 注意不用 \setlength
\renewcommand{\footrulewidth}{0pt}

% % 自定义 myfancy 样式
\fancypagestyle{myfancy}{%
	\fancyhf{}
	\fancyhead{...}
	\fancyfoot{...}
}
% 使用样式
%\pagestyle{myfancy}

%opening
\title{}
\author{}

\begin{document}

	\section{}


	第五章 排版样式设定

% 字号、斜体

		{\small The small and
	\textbf{bold} Romans ruled}
		{\Large all of great big
		{\itshape Italy}.}

% \usepackage[T1]{fontenc} 
% => fontenc 宏包是用来配合传统的 LATEX 字体的

% 使用 fontspec 宏包调用 ttf 或 otf 格式字体
% fontspec 宏包会覆盖数学字体设置。需要调用表 5.4 中列出的一些数学字体宏包时,应当在调用 fontspec 宏包时指定 no-math 选项
% fontspec 宏包可能被其它宏包或文档类(如 ctex 文档类)自动调用时,则在文档开头的 \documentclass 命令里指定 no-math 选项


% \setCJKmainfont{⟨font name⟩}[⟨font features⟩]
% \setCJKsansfont{⟨font name⟩}[⟨font features⟩]
% \setCJKmonofont{⟨font name⟩}[⟨font features⟩]

% 使用 unicode-math 宏包配置 Unicode 数学字体

% 强调 = 下划线?
	Some \emph{emphasized words,
		including \emph{double-emphasized}
		words}, are shown here.

	An example of \uline{some long and underlined words.}

% 行距 \linespread{⟨factor⟩}
% 其中 ⟨factor⟩ 作用于基础行距而不是字号。缺省的基础行距是 1.2 倍字号大小(参考 \font-size 命令)
%,因此使用 \linespread{1.5} 意味着最终行距为 1.8 倍的字号大小

	{\linespread{2.0}\selectfont
	The baseline skip is set to be
	twice the normal baseline skip.
	Pay attention to the \verb|\par|
	command at the end. \par}
	In comparison, after the
	curly brace has been closed,
	everything is back to normal

% 段落格式
	\setlength{\parskip}{1ex plus 0.5ex minus 0.2ex}
	In comparison, after the
	curly brace has been closed,
	everything is back to normal

% 水平间距
	见过 \quad 和 \qquad 命令,它们也可以用于文本中,分别相当于
	\hspace{1em} 和 \hspace{2em}

% 垂直间距

	A paragraph.

	\vspace{2ex}
	Another paragraph.

	Use command \verb|\vspace{12pt}|
	to add \vspace{12pt} some spaces
	between lines in a paragraph.

	Or you can use \verb|\\[12pt]|
	to \\[12pt] add vertical space,
	but it also breaks the line.

% 页面内容对齐的考虑
% LATEX 默认将页面内容在垂直方向分散对齐。对于有大量图表的文档,许多时候想要做
% 到排版匀称的页面很困难,垂直分散对齐会造成某些页面的垂直间距过宽,还可能报大量的
%Underfull \vbox 警告
% 以下命令分别令页面在垂直方向向顶部对齐 / 分散对齐:
% \raggedbottom
% \flushbottom

% 页眉页脚样式
%\pagestyle{⟨page-style⟩}

% fancyhdr 改善页眉页脚内容样式

\end{document}
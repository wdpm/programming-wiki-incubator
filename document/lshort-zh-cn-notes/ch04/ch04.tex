\documentclass[]{ctexart}

\usepackage{amsmath}
\usepackage{amssymb}
\usepackage{fontspec}
\usepackage{amsthm}
\usepackage[a4paper, left=2cm, right=2cm, top=2cm, bottom=2cm]{geometry}

%opening
\title{}
\author{}

\begin{document}

	\section{}

	第四章

	The Pythagorean theorem is
	$a^2 + b^2 = c^2$

	\iffalse
	单独成行的行间公式在 LATEX 里由 equation 环境包裹。equation 环境为公式自动生成一
	个编号,这个编号可以用 \label 和 \ref 生成交叉引用,amsmath 的 \eqref 命令甚至为引用
	自动加上圆括号;还可以用 \tag 命令手动修改公式的编号,或者用 \notag 命令取消为公式编
	号(与之基本等效的命令是 \nonumber)
	\fi

	The Pythagorean theorem is:
	\begin{equation}
		a^2 + b^2 = c^2 \label{pythagorean}
	\end{equation}
	Equation \eqref{pythagorean} is
	called `Gougu theorem' in Chinese.

% 修改公式编号

	It's wrong to say
	\begin{equation}
		1 + 1 = 3 \tag{dumb}
	\end{equation}

% 取消公式编号
	or
	\begin{equation}
		1 + 1 = 4 \notag
	\end{equation}

% 行间公式
	\begin{equation*}
		a^2 + b^2 = c^2
	\end{equation*}
	For short:
	\[a^2 + b^2 = c^2 \]
	Or if you like the long one:
	\begin{displaymath}
		a^2 + b^2 = c^2
	\end{displaymath}

% $...$ 是行内公式,非常窄,排版空间很小

	In text:
	$\lim_{n \to \infty}
	\sum_{k=1}^n \frac{1}{k^2}
	= \frac{\pi^2}{6}$.

% 行间公式周围空间很宽裕,推荐

	In display:
	\[
		\lim_{n \to \infty}
		\sum_{k=1}^n \frac{1}{k^2}
		= \frac{\pi^2}{6}
	\]

% mathbb 在公式中输入正文符号 R

	$x^{2} \geq 0 \qquad
	\text{for \textbf{all} }
	x\in\mathbb{R}$

% 在矩阵中可能会用到竖排的 ... (\vdots) 和斜排的 ... (\ddots)

	$a_1, a_2, \dots, a_n$ \\
	$a_1 + a_2 + \cdots + a_n$

% 指数、上下标和导数

	$p^3_{ij} \qquad
	m_\mathrm{Knuth}\qquad
	\sum_{k=1}^3 k $\\[5pt]
	$a^x+y \neq a^{x+y}\qquad
	e^{x^2} \neq {e^x}^2$

	$f(x) = x^2 \quad f'(x)
	= 2x \quad f''^{2}(x) = 4$

%  分式
	In display style:
	\[
		3/8 \qquad \frac{3}{8}
		\qquad \tfrac{3}{8}
	\]
	In text style:
	$1\frac{1}{2}$~hours \qquad
	$1\dfrac{1}{2}$~hours

% sqrt
	$\sqrt{x} \Leftrightarrow x^{1/2}
	\quad \sqrt[3]{2}
	\quad \sqrt{x^{2} + \sqrt{y}}$

% 二项式
	Pascal's rule is
	\[
		\binom{n}{k} =\binom{n-1}{k}
		+ \binom{n-1}{k-1}
	\]

% 关系符号
	\[
		f_n(x) \stackrel{*}{\approx} 1
	\]

% 算法举例
	\[
		\lim_{x \rightarrow 0}
		\frac{\sin x}{x}=1
	\]


	$a\bmod b \\
	x\equiv a \pmod{b}$

% 巨算符

	In text:
	$\sum_{i=1}^n \quad
	\int_0^{\frac{\pi}{2}} \quad
	\oint_0^{\frac{\pi}{2}} \quad
	\prod_\epsilon $ \\

	In display:
	\[\sum_{i=1}^n \quad
	\int_0^{\frac{\pi}{2}} \quad
	\oint_0^{\frac{\pi}{2}} \quad
	\prod_\epsilon \]

% 巨算符的上下标位置可由 \limits 和 \nolimits 调整,前者令巨算符类似 lim 或求和算符
%�,上下标位于上下方;后者令巨算符类似积分号,上下标位于右上方和右下方
	In text:
	$\sum\limits_{i=1}^n \quad
	\int\limits_0^{\frac{\pi}{2}} \quad
	\prod\limits_\epsilon $ \\
	In display:
	\[\sum\nolimits_{i=1}^n \quad
	\int\limits_0^{\frac{\pi}{2}} \quad
	\prod\nolimits_\epsilon \]

% \usepackage{amssymb}
	\[
		\sum_{\substack{0\le i\le n \\
		j\in \mathbb{R}}}
		P(i,j) = Q(n)
	\]
	\[
		\sum_{\begin{subarray}{l}
				  0\le i\le n \\
				  j\in \mathbb{R}
		\end{subarray}}
		P(i,j) = Q(n)
	\]

% 数学重音和上下括号
	$\bar{x_0} \quad \bar{x}_0$\\[5pt]
	$\vec{x_0} \quad \vec{x}_0$\\[5pt]
	$\hat{\mathbf{e}_x} \quad{\tiny}
	\hat{\mathbf{e}}_x$

	$0.\overline{3} =
	\underline{\underline{1/3}}$ \\[5pt]
	$\hat{XY} \qquad \widehat{XY}$\\[5pt]
	$\vec{AB} \qquad
	\overrightarrow{AB}$

	$\underbrace{\overbrace{(a+b+c)}^6
	\cdot \overbrace{(d+e+f)}^7}
	_\text{meaning of life} = 42$

% 长度伸长的箭头
	\[a\xleftarrow{x+y+z} b \]
	\[c\xrightarrow[x<y]{a*b*c}d \]

% \left 和 \right 必须成对使用。需要使用单个定界符时,另一个定界符写成 \left. 或 \right.
	\[1 + \left(\frac{1}{1-x^{2}}
	\right)^3 \qquad
	\left.\frac{\partial f}{\partial t}
	\right|_{t=0}\]

% 长公式折行
	\begin{multline}
		a + b + c + d + e + f
		+ g + h + i \\
		= j + k + l + m + n\\
		= o + p + q + r + s\\
		= t + u + v + x + z
	\end{multline}

	\begin{align}
		a & = b + c \\
		& = d + e
	\end{align}

	\begin{align}
		a ={} & b + c \\
		={} & d + e + f + g + h + i
		+ j + k + l \notag \\
		& + m + n + o \\
		={} & p + q + r + s
	\end{align}

	\begin{align}
		a &=1
		&
		b &=2
		& c &=3
		\\
		d &=-1 &
		e &=-2
		& f &=-5
	\end{align}

% 不对齐,仅罗列
	\begin{gather}
		a = b + c \\
		d = e + f + g \\
		h + i = j + k \notag \\
		l + m = n
	\end{gather}

% 数学公式的尺寸,可以通过 displaystyle 这类指令来控制
	\[
		r = \frac
		{\sum_{i=1}^n (x_i- x)(y_i- y)}
		{\displaystyle \left[
			\sum_{i=1}^n (x_i-x)^2
			\sum_{i=1}^n (y_i-y)^2
			\right]^{1/2} }
	\]

	\theoremstyle{definition} \newtheorem{law}{Law}
	\theoremstyle{plain} \newtheorem{jury}[law]{Jury}
	\theoremstyle{remark} \newtheorem*{mar}{Margaret}

	\begin{law}
		\label{law:box}
		Don't hide in the witness box.
	\end{law}
	\begin{jury}[The Twelve]
		It could be you! So beware and
		see law~\ref{law:box}.
	\end{jury}
	\begin{jury}
		You will disregard the last
		statement.
	\end{jury}
	\begin{mar}
		No, No, No
	\end{mar}
	\begin{mar}
		Denis!
	\end{mar}

% 一个编号

	\begin{equation}
		\begin{aligned}
			a &= b + c \\
			d &= e + f + g \\
			h + i &= j + k \\
			l + m &= n
		\end{aligned}
	\end{equation}

% 数组和矩阵
	\[\mathbf{X} = \left(
	\begin{array}{cccc}
		x_{11} & x_{12} & \ldots & x_{1n} \\
		x_{21} & x_{22} & \ldots & x_{2n} \\
		\vdots & \vdots & \ddots & \vdots \\
		x_{n1} & x_{n2} & \ldots & x_{nn} \\
	\end{array} \right)\]

	\[ |x| = \left\{
	\begin{array}{rl}
		-x & \text{if} x < 0, \\
		0  & \text{if} x = 0, \\
		x  & \text{if} x > 0.
	\end{array} \right. \]

% cases 等价于上面的 array 排版
	\[ |x| =
	\begin{cases}
		-x & \text{if} x < 0,\\
		0 & \text{if} x = 0,\\
		x & \text{if} x > 0.
	\end{cases} \]

	\iffalse
	我们当然也可以用 array 环境排版各种矩阵。amsmath 宏包还直接提供了多种排版矩阵的
	环境,包括不带定界符的 matrix,以及带各种定界符的矩阵 pmatrix(�)、bmatrix(�)、Bmatrix
	(�)、vmatrix(
	��)、Vmatrix(
	��)。使用这些环境时,无需给定列格式
	\fi

	\[
		\begin{matrix}
			1 & 2 \\ 3 & 4
		\end{matrix} \qquad
		\begin{bmatrix}
			x_{11} & x_{12} & \ldots & x_{1n} \\
			x_{21} & x_{22} & \ldots & x_{2n} \\
			\vdots & \vdots & \ddots & \vdots \\
			x_{n1} & x_{n2} & \ldots & x_{nn} \\
		\end{bmatrix}
	\]

% 间距修复
	\[
		\int_a^b f(x)\mathrm{d}x
		\qquad
		\int_a^b f(x)\,\mathrm{d}x
	\]

% 多重积分号。如果我们直接连写两个 \int,之间的间距将会过宽,此时
% 可以使用负间距 \! 修正之
	\newcommand\diff{\,\mathrm{d}}
	\begin{gather*}
		\int\int f(x)g(y)
		\diff x \diff y \\
		\int\!\!\!\int
		f(x)g(y) \diff x \diff y \\
		\iint f(x)g(y) \diff x \diff y \\
		\iint\quad \iiint\quad \idotsint
	\end{gather*}

% 数学字母字体,举例
	$\mathbf{ABC}$

% 加粗的数学符号
	$\mu, M \qquad
	\boldsymbol{\mu}, \boldsymbol{M}$

\end{document}